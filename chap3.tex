\chapter{Ant-Based Community Detection}
\begin{spacing}{1.5}
In this chapter, we describe our ant-based approach for detecting communities in complex networks. The algorithm is divided into three main phases: initialization, exploration and construction. The construction phase is followed by a local optimization step to enhance the solution obtained initially.

\section{Overview}

Since our algorithm is ant-based, it consists of artificial ants which explore the graph based on a set of rules. In each cycle, all of the ants explore a local section of the graph by choosing edges based on their pheromone level. Edges with a higher pheromone value are more likely to be chosen and the pheromone on an edge is marked to be increased if it's chosen. Based on the rules specified for exploration, ants try to discover intracommunity edges in their local section of the graph and as a result we expect them to have a higher value of pheromone towards the end of the exploration phase. The basic outline of the algorithm for the exploration phase is derived from~\cite{5910378}.  \\
\indent Once the exploration phase is complete, we examine the edges with the highest pheromone levels and use them to obtain an intial clustering of the graph. After this, a local optimization step is performed which involves reassigning node communities and possibly merging clusters before returning the final partitioning. An outline of the algorithm is shown in Figure~\ref{fig1}.
\SetKwInput{Input}{Input}
\SetKwFunction{antcommunity}{AntCommunity}
\SetKwInput{Output}{Output}
\SetKwFunction{initants}{InitAnts}
\SetKwFunction{antsmove}{AntsMove}
\SetKwFunction{firstpart}{PartitionOneLevel}
\SetKwFunction{reset}{ResetAnts}
\SetKwFunction{reassign}{ReassignClusters}

\begin{figure}
\label{fig1}
\hrule
\BlankLine
\begin{algorithm}[H]
	\SetAlgoLined
	\DontPrintSemicolon
	\antcommunity{$G$} \;
	\Input{$G=(V,E)$, graph whose community structure is to be found }
	\Output{Set of $k$ communities $C = \{C_1, \ldots, C_k\}$}
	\Begin{
		//Initialization \;
		$i \leftarrow 1$ \;
		\ForEach{$v \in V$}{
			Calculate the number of common nodes with all neighbors \;
		}
		Create the set of ants $A$, of size $|V|$ \;
		\initants{$G, A$}\;
		\BlankLine
		//Exploration phase \;
		\While{$i < i_{max}$}{
			\antsmove{$G, A$} \; 
			Update phermone evaporation factor $\eta$ \;
			\reset{$G, A$} \;
			$i \leftarrow i + 1$ \;
		}
		\BlankLine		
		//Construction phase \;
		Sort $E$ in decreasing order of pheromone \;
		\firstpart{$G$} \;
		\reassign{} \;
	}

\end{algorithm}
\BlankLine
\begin{algorithm}[H]
	\SetAlgoLined
	\DontPrintSemicolon
	\initants{G, A} \;
	\Begin {
		\For{$i=0; i < |V|; i++$}{
			$A[i].location \leftarrow V[i]$ \;
			$A[i].tabuList \leftarrow \varnothing$ \;
		}
		\ForEach{$e \in E$}{
			$e.initPhm = 1$ \;
			$e.nVisited = 0$ \;
		}
	}
\end{algorithm}
\BlankLine
\hrule
\caption{Ant-Based Community Detection Algorithm}
\end{figure}

\section{Data Structures}

Here we will describe the data structures used in the algorithm to facilitate the description of the algorithm. The main data structures in the algorithm are: the graph, weighted graph and ants. The graph consists of a vertex set $V$ and an edge set $E$ where each edge is augmented with pheromone information ($phm$), the number of times it has been traversed in a cycle ($num\_visited$) and initial pheromone level ($init\_phm$). Each vertex is also augmented with information regarding how many nodes are adjacent to it ($neighbors$) and the number of nodes it has in common with each of its neighbors ($common$).\\
\indent The weighted graph $G_w$ represents a partioning of the original graph. It is also divided into a vertex set $V_w$ and an edge set $E_w$. Each vertex in $V_w$ represents a cluster and stores additional information such as the sum of the total pheromone amongst all the edges in the cluster ($weight$), the number of edges falling within that cluster ($in\_links$), the total pheromone of that cluster ($total$) which represents its $weight$ plus the pheromone along each of its edges and a list of the clusters adjacent to it. Since there can be multiple edges between between two different clusters from the original graph, all such edges are collapsed into one in $G_w$, hence each edge in $G_w$ keeps track of the total number of edges falling between the two clusters from the original graph ($cross\_edges$) and the total pheromone between the two clusters ($cross\_phm$).\\
\indent The set of ants $A$ has a fixed cardinality, $|V|$. Each ant maintains its current location ($location$) which is a vertex and a tabu list which stores the most recently vertices visited.

\section{Initialization}

Since each vertex of the weighted graph represents a cluster, we have defined a parameter ($threshold$) for each vertex in $G_w$ which determines if a cluster is well connected or should be considered for merging with another cluster during the optimization phase. The threshold value is set based on the percentage of total possible edges the graph has and was chosen experimentally.\\
\indent In the initilization phase we first set the threshold value depending on the size of the network whose community structure is to be found. After this we create the sets of ants $A$ and place each ant on a vertex of the graph. The initial pheromone level and $num\_visited$ for each edge is set to 1 and 0 respectively by default. This is done so that in the beginning of the algorithm all the edges have an equal chance of being selected, so it encourages exploration.\\
\indent After the above step, we calculate the number of nodes each vertex has in common with its neighbors. Since each vertex maintains a list of neighbors, the number of common nodes can be computed by a simple set intersection operation.

\section{Exploration}

In this phase, the ants explore the graph and lay pheromone along the edges. The aim is to discover edges whose vertices are densely connected. Since communities can be intuitively thought of subgroups of vertices who share more vertices in common with respect to other, this means that two adjacent vertices are more likely to be in the same community if they share more number of neighbors. This is the rule that the ants follow to discover such edges. The exploration phase is shown in Figure~\ref{fig2}.\\
\SetKwFunction{update}{UpdatePheromone}
\begin{figure}
\hrule
\BlankLine
\label{fig2}
\begin{algorithm}[H]
	%\SetAlgoLined
	\DontPrintSemicolon
	\antsmove{$G,A$} \;
	\Begin {
		\For{$s=1$ \KwTo $maxSteps$}{
		\If{$s \mod updatePeriod == 0$}{
			\update{G} \;
		}
		\ForEach{$a \in A$}{
			$nAttempts \leftarrow 0$ \;
			$moved \leftarrow$ False \;
			\While{not $moved$ and $nAttempts < 5$}{
				$v_1 \leftarrow a.location$ \;
				Select an edge $(v_1, v_2)$ at random and proportional to the
				phermone level and the size of their neighborhood overlap \;
				\eIf {$v_2 \notin a.tabuList$}{
					add $v_2$ to $a.tabuList$ \;
					$a.location \leftarrow v_2$ \;
					$(v_1, v_2).nVisited++$ \;
					$moved \leftarrow$ True \;
				} {
					$nAttempts++$ \;
				}
			}
		}
	}
	}
	\end{algorithm}

	\BlankLine

	\begin{algorithm}[H]
	\SetAlgoLined
	\DontPrintSemicolon
	\update{G}
	\Begin {
		\ForEach{$e \in E$} {
			$e.phm \leftarrow (1 - \eta) \times e.phm + e.nVisited \times   e.initPhm$ \;
			$e.nVisited \leftarrow 0$ \;
		}
		\If{$e.phm < minPhm$}{
			$e.phm = minPhm$ \;
		}
	}
\end{algorithm}
\hrule
\BlankLine
\caption{Exploration phase}
\end{figure}
\indent In each iteration, all the ants are moved in parallel on the graph for a fixed number of steps. This is continued until the maximum number of iterations is exceeded. For purposes of efficiency, it is better to update the pheromone after a fixed number of steps instead of updating it after every step.\\
\indent As mentioned previously, the movement of the ants is determined by the pheromone level of the edges incident to the current vertex and the number of nodes the current vertex has in common with its neighbors (neighborhood overlap). This is also known as proportional selection. The edges connecting vertices in the same community will be more likely to be selected by the algorithm and as a result they will tend to have a higher pheromone value. \\
\indent When an ant traverses an edge, it is scheduled to be increased by an amount equal to its initial pheromone value~\cite{5910378}. Since the pheromone on all edges is initialized to 1, when the pheromone update is performed, the pheromone on each edge is increased by the number of times it was traversed.\\
\indent Proportional selection is performed as follows. For each neighboring node, the ant calculates the sum of the pheromone along the edge and the number of nodes in common with that neighbor and stores it in an array whose size is equal to the degree of the current vertex. While filling up the array we also maintain a running sum of the total, let this be denoted by $sum$. Then a random number between $[0, sum]$ is generated. Then we sum up the array elements until the random value is reached. The index of this element corresponds to the index of the neighbor in the adjacency list and the ant chooses to move to that vertex. This way the ants will have a higher probability of choosing an edge connecting a vertex in the same community. When ant traverses an edge, it adds the chosen vertex to its tabu list to avoid visiting it multiple times.\\
\indent We employ a couple of mechanisms to avoid getting caught in local optima. As mentioned previously, each ant maintains a tabu list of a fixed size and are prohibited from choosing vertices already present in the list. Also, the pheromone level of each edge is evaporated periodically by a certain factor which reduces over time. In the beginning, the evaporation rate is higher so as to encourage exploration and it is gradually decreased. The minimum pheromone level is set to 1 as due to evaporation we don't want the pheromone level of an edge to become so low that it may never be considered again, this also avoids the ants from converging to a small set of edges.\\
\indent The pheromone update is performed periodically, for reasons of efficiency. As mentioned previously, each edge keeps a track of the number of times it has been traversed since the previous update. This information is used along with the current pheromone level for the update. The formula is that mentioned in~\cite{5910378}:

\begin{align}
e.phm = (1 - \eta)e.phm + e.num\_visited \times e.init\_phm
\end{align}

where $e$ is the edge in the graph being updated, $phm$ is the pheromone level of $e$, $\eta$ is the evaporation rate, $nVisited$ is the number of times the edge was traversed since the last update cycle and $initPhm$ is the initial pheromone level, in this case $initPhm = 1, \forall e$.\\
\indent At the end of each cycle, two operations are performed. First, the pheromone evaporation rate is updated by a constant factor. This is to aid exploration in the beginning by having a large evaporation rate and reducing it so that the ants slowly begin to converge on a set of edges. The initial value of $\eta$ is set to 0.5 and it is multiplied by 0.95 at the end of every iteration. Second, the ants are reset before the start of the next iteration. About half the ants stay in their current location while the other half are randomly placed as shown in Figure~\ref{fig3} The maximum number of iterations possible during each step of the exploration phase is 75.
\begin{figure}
\label{fig3}
\hrule
\BlankLine
\begin{algorithm}[H]
	\SetAlgoLined
	\DontPrintSemicolon
	\reset{$G, A$} \;
	\Begin {
		\ForEach{$a \in A$}{
		\If{$Random(0, 1) < 0.5$}{
			$a.location \leftarrow Random(0, |V| - 1)$ \;
			}
		$a.tabuList \leftarrow \varnothing$ \;
		}
	}
\end{algorithm}
\BlankLine
\hrule
\caption{Reset ants}
\end{figure}

\section{Construction}

At the end of the exploration phase we expect the edges falling within different communities to have a higher pheromone level. The construction phase utilizes this information to partition the graph. The output of the construction phase is a new graph called the weighted graph. Each community corresponds to a vertex of this graph.\\
\indent First, the edges of the graph are sorted in decreasing order of pheromone. An array called $nodeToComm$ (of size $|V|$) stores the community membership of each node. It is initialized to -1 implying the node has not been assigned a cluster. A variable, $comm$, initialized to 0, keeps track of the number of communities found. \\
\indent To build the weighted graph we read in each edge $(i, j)$ sequentially. If both $i$ and $j$ are not assigned a community i.e.\ $node2comm[i]$ and $node2comm[j]$ are both -1 then they are assigned a new cluster and $comm$ is incremented. If one of $i$ and $j$ is assigned a community but the other isn't, we add it to that community. If both $i$ and $j$ are in separate communities, i.e.\ $nodeToComm[i] \neq nodeToComm[j]$, then we create the edge $(nodeToComm[i], nodeToComm[j])$ in the weighted graph if it doesn't exist and update the number of crossing (intercommunity) edges in the new graph.\\
\indent While creating this first partition, each node (of the original graph) tracks its internal degree, the number of connections it has to nodes in its own community, and external degree. This information is used in the next phase for optimizing the current solution. \\
\indent At the end of the construction phase we obtain an initial partition of the graph using the process described above. This weighted graph is now optimized by reassigning nodes which may have been correctly placed in the wrong cluster and then merging clusters to form more cohesive communities.\\

\section{Optimization Phase}

As mentioned before, we expect a node to have a higher number of connections to nodes inside its community. Since the construction phase only utilizes pheromone infomation to cluster the graph, it is likely that nodes are assigned clusters to which they aren't well connected. We can utilize the internal and external degrees computed for each vertex to correct these mistakes.\\
\indent First, the nodes are sorted in decreasing order of external degree. Then, we for each node in the sorted order we determine the cluster to which the node has the highest external degree and compare that with its internal degree. If the external degree to that cluster is greater, then the nodes cluster is reassigned to that one. As nodes get reassigned, the internal and external degrees of the nodes to which they are connected might change but it is too costly to constantly update this information and redo the assigning. As a result, once the reassigning step is completed, we rebuild the weighted graph from the cluster assignments and recompute the internal and external degrees of each node. Then the reassignment step is repeated. \\
\indent In order to determine the termination point for this step, we calculate the modularity of the partition obtained after reassigning and compare it with the modularity obtained at the previous step, if there is an increase in modularity then we choose that as the current best partition and repeat the steps until there is no improvement in modularity. To avoid getting caught in local optima, we allow the modularity for each successive parititions to decrease for a fix number of steps, specified by a parameter ($max\_decrease$). If this threshold is exceeded then we terminate this step and return the best modularity paritition.\\
\indent Since the ants are used to discover intracommunity edges, we can use the pheromone level in the weighted graph to merge clusters. The communities in the weighted graph so far may not be cohesive. Since we expect a community to have high density of internal links as compared to external links, this means that if a vertex has a high $weight$ as compared to its total pheromone, then its a good community. Otherwise, its possible that the pheromone along one of the intercommunity edges will be higher as compared to its $weight$ and we can merge the two communities to obtain a better solution. Since intercommunity edges from the original graph are condensed to one edge in the weighted graph, the more intercommunity edges we have, the higher we expect the pheromone to be along the corresponding edge in the weighted graph.\\
\indent The fraction of the total pheromone along each outgoing edge for a vertex is calculated. Since the graph is undirected, the fraction is calculated using the total contribution from both endpoints. The fractional values for each edge are stored in an array and sorted in decreasing order. For an edge with a high fractional value, it means that the two clusters have a high number of connections between them and a better solution would be obtained by merging them.\\
\indent In the last step, the edges sorted in decreasing order of fractional values are used to merge different clusters. For each edge $(u, v)$, if the fraction of the total pheromone on $(u,v)$ is more than the fraction of the total pheromone in vertex $u$ or $v$, they are merged. Else if more than half of the total pheromone lies within the both vertices $u$ and $v$, then we don't merge them.\\

\section{Parameters}

The various parameters in the algorithm are mentioned in Table~\ref{tab1}. These parameters aren't for a single type of graph but have been used for all graphs on which the algorithm is tested.

\begin{table}
\caption{Parameters in the algorithm}
\label{tab1}
\BlankLine
\begin{tabular}{ | c | c | l | }
	\hline
	\textbf{Parameter} & \textbf{Value} & \textbf{Comments} \\
	\hline
	$i_{max}$ & 75 & Maximum number of iterations \\
	\hline
	$maxSteps$ & $\frac{1}{3} |V|$ or 75 & Maximum number of steps in each iteration \\
	\hline
	$\eta$ & 0.5 & Pheromone evaporation rate \\
	\hline
	$\Delta \eta$ & 0.95 & Pheromone update constant \\
	\hline
	$updatePeriod$ & $maxSteps / 3$ & Number of cycles between pheromone update \\
	\hline
	$LIST\_SIZE$ & 2 or 5 & Tabu list size\\
	\hline
\end{tabular}
\end{table}
\end{spacing}