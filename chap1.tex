\chapter{Introduction}
\begin{spacing}{1.5}
\thispagestyle{empty}
Complex networks are extensively used to model various real-world systems such as social networks, technological (Internet and World Wide Web) networks, biological networks etc. These networks are modeled as graphs where nodes represent the objects in the system and edges represent the relationship among these objects. For example, in a social network, nodes can represent people and two nodes are connected by a link if they are friends with each other. \\
\indent These networks exhibit distinctive statistical properties. The first property is the ``small world effect'', which implies that the average distance distance between vertices in a network is short~\cite{milgram67smallworld}. The second is that the degree distributions follow a power-law~\cite{Barabasi99emergenceScaling}, and the third one is network transitivity which is the property that two vertices who are both neighbors of the same third vertex, have an increased probabilibty of being neighbors of one another~\cite{Watts-Colective-1998}.\\
\indent Another property which appears to be common to such networks is that of community structure (or clustering). While the concept of a community is not strictly defined in the literature as it can be affected by the application domain, one intuitive notion of a community is that it consists of a subset of nodes from the original graph which between them have a higher density of links as compared to their links with the rest of the graph. In this thesis, we describe an ant-based algorithm for automatically detecting communities in graphs, without specifying the number of communities beforehand.\\
\indent Over the course of more than a decade, the task of finding communities in networks has received enormous attention from researchers in different fields such as physics, statistics, computer science etc. As a result, there are currently a vast number of methods which can be used to evaluate the community structure of a network. These methods are described in the next chapter.\\
\indent Ant algorithms have been previously used to detect communities in graphs~\cite{DBLP:journals/corr/abs-1303-4711}~\cite{5586496}~\cite{Jin:2011:ACO:2022850.2022861}. In our approach, we use artificial ants which traverse the graph based solely on local information and deposit pheromone as they travel. This algorithm uses the cumulative pheromone on the edges to build up an initial clustering of the graph. Then a local optimization method is used to reassign the clusters of different nodes based on their degree distribution after which clusters are merged depending on certain rules to obtain the final partitioning of the graph.\\
\indent The rest of this thesis is organized as follows. Chapter 2 provides more detailed information about the problem statement and covers the previous work done. The ant-based algorithm is described in Chapter 3. Chapter 4 covers metrics to evaluate partitions and Chapter 5 covers the performance of the algorithm on various synthetic and real-world graphs and compares it to existing algorithms. The conclusion is given in Chapter 6.
\end{spacing}